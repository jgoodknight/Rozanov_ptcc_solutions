
\chapter{Three Important Probability Distributions}

\section{Important/Useful Theorems}

\subsection{The Binomial Distribution}
For $n$ trials where the probability of success is $p$ and the probability of failure is $q=1-p$, the probability of their being $k$ successes is:
\begin{equation}
	p(k) = \binom{n}{k} p^k q^{n-k}
\end{equation}

\subsection{The Poisson Distribution}
For $n$ trials where the probability of success is $p$ and is very small and the number of trials is very large, the probability of their being $k$ successes is:
\begin{equation}
	p(k) = \frac{a^k}{k!}e^{-a}
\end{equation}
Where $a=np$, the average number of successes.

\subsection{The Normal Distribution}
As $n \rightarrow \infty$, the Binomial Distribution goes to:
\begin{equation}
	p(x) = \frac{1}{\sqrt{2 \piu \sigma^2}}e^{\frac{-(x-a)^2}{2 \sigma^2}}
\end{equation}


\section{Answers to Problems}


\subsection{}
%problem5.1
If you call tails, there are two possibilities for the outcome of the coin-toss experiment which, since they are not correlated to your guess, are both one-half; similarly for heads.  Therefore, no matter what you do, the probability of guessing correctly is one-half.

If, however, the coin is biased towards heads, there is a probability greater than one half that you'll win the trial if you call heads; therefore you should always call heads for a coin biased towards heads and similarly always call tails for a coin biased towards tails.

\end{equation}
\textbf{Answer not verified}


\subsection{}
%problem5.2

In order to avoid any blatant sexism, we shall define  ``success'' as having a boy... What?  Anyway, for a couple having $n$ children, the probability of having $k$ boys will clearly follow a binomial distribution:

\begin{equation}
	p_n(k) = \binom{n}{k} \frac{1}{2^n}
\end{equation}
\subsubsection{a}
\begin{equation}
	p_{10}(5) = \binom{10}{5} \frac{1}{2^{10}} = \frac{63}{256} = 0.246094
\end{equation}
\subsubsection{b}
\begin{equation}
	p_{10}(3\rightarrow 7) = \sum_{i = 3}^7 \binom{10}{i} \frac{1}{2^{10}} = \frac{57}{64} = 0.890625
\end{equation}
\textbf{Answer not verified}


\subsection{}
%problem5.3

Since $p=.001$ and $n=5000$ we can cautiously use the Poisson Distribution where $a=5$.  Since we want to know the probability of hitting 2 or more, we will take the probability of hitting either 0 or 1 and subtract from 1.

\begin{equation}
	p(k \geq 2) = 1 - p(0) - p(1) \approx 1 - \frac{5^0}{0!}e^{-5} - \frac{5^1}{1!}e^{-5} = 1 - 6e^{-5} \approx 0.96
\end{equation}

\textbf{Answer verified}

\subsection{}
%problem5.4

Again we cautiously use the Poisson Distribution since $p=\frac{1}{500}$ and $n=500$ and again use the same trick as before to get the chance of greater than two ``successes''.
\begin{equation}
	p(k \geq 2) = 1 - p(0) - p(1) \approx 1 - \frac{1^0}{0!}e^{-1} - \frac{1^1}{1!}e^{-1} = 1 - \frac{2}{e} \approx 0.264241
\end{equation}
Note that the book has most definitely got this problem wrong as:
\begin{equation}
	p(k \geq 3) = 1 - p(0) - p(1) - p(2) \approx 1 - \frac{1^0}{0!}e^{-1} - \frac{1^1}{1!}e^{-1}- \frac{1^2}{2!}e^{-1} = 1 - \frac{5}{2e} \approx .0803
\end{equation}

\textbf{Answer verified-ish}


\subsection{}
%problem5.5

Without any special cases, this problem is the good-old binomial distribution:

\begin{equation}
	p(k) = \binom{n}{k} p^k(1-p)^{n-k}
\end{equation}
So we simply sum up all the even probabilities:
\begin{equation}
	P_n = \sum_{i=0}^{\frac{n}{2}} p(2i) = \sum_{i=0}^{\frac{n}{2}} \binom{n}{2i} p^{2i}(1-p)^{2i-k}
\end{equation}
Which in the total lack of any skill in mathematics, I plug into Mathematica to get:
\begin{equation}
	P_n = \frac{1}{2} \left((1-2 p)^n+1\right)
\end{equation}

\textbf{Answer verified}

\subsection{}
%problem5.6
In an infinite series of Bernouli trials, let the event that the $i^{th}$ 3-tuple has the pattern SFS be $A_i$ and $P(A_i)=p_i$.  Since we've designed these tuples to not overlap, these trials are independent such that:
\begin{equation}
	P\left( \bigcup_{i=1}^{\infty} A_i \right) = \sum _{i=1}^{\infty} p_i \rightarrow \infty
\end{equation}
By the second Borel-Cantelli lemma.

\textbf{Answer not verified}

\subsection{}
%problem5.7

WE want the probability of 3 or more very rare events happening in a large number of trials: this is a Poisson distribution with $a=1000\cdot0.001=1$
\begin{equation}
	p(k \geq 3) = 1 - p(0) - p(1) - p(2) \approx 1 - \frac{1^0}{0!}e^{-1} - \frac{1^1}{1!}e^{-1}- \frac{1^2}{2!}e^{-1} = 1 - \frac{5}{2e} \approx .0803
\end{equation}

\textbf{Answer not verified}

\subsection{}
%problem5.8

We'll go ahead and call $p=\frac{1}{365}$ to be small and 730 to be big so that we can do Poisson with $a=2$.
\begin{equation}
	p(2) = \frac{2^2}{2!}e^{-2} = \frac{2}{e^2} \approx 0.270671
\end{equation}

\textbf{Answer not verified}

\subsection{}
%problem5.9

\begin{equation}
	\textbf{E}\xi = \sum_{k=0}^{\infty} k \frac{a^k}{k!}e^{-a} = e^{-a}a\sum_{k=0}^{\infty} \frac{a^{(k-1)}}{(k-1)!} = a
\end{equation}
\begin{equation}
	\textbf{E}\xi^2 = \sum_{k=0}^{\infty} k^2 \frac{a^k}{k!}e^{-a} = a^2 + a
\end{equation}
That last one I got lazy and used Mathematica, sorry!  Clearly, though $\sigma^2 = a$.
\begin{eqnarray}
	\frac{\textbf{E}(\xi - a)^3}{\sigma^3} = \frac{\textbf{E}(\xi^3 - 3\xi^2a + 3\xia^2 - a^3)}{a^{\frac{3}{2}}} \\
	 \textbf{E}\xi^3 = a^3+3 a^2+a \\
	 \textbf{E}(\xi^3 - 3\xi^2a + 3\xi a^2 - a^3) = (a^3+3 a^2+a) - 3a(a^2 + a) + 3a^3 - a^3 = a \\
	 \frac{\textbf{E}(\xi - a)^3}{\sigma^3} = \frac{a}{a^{\frac{3}{2}}} = \frac{1}{\sqrt{a}}
\end{eqnarray}
\textbf{Answer verified}

\subsection{}
%problem5.10

\textbf{SKIPPED}

\subsection{}
%problem5.11
Finally! something other than the Poisson distribution... back to... NORMAL!  Ha!
\begin{eqnarray}
	\textbf{E}x = np = 30 \\
	\textbf{D}x = npq = 30\cdot .7 = 21 \\
	p = \int_{20}^{40} \frac{1}{\sqrt{2 \pi 21}} e^{\frac{-(x-30)^2}{2 \cdot 21}} \approx 0.970904
\end{eqnarray}
\textbf{Answer verified}


\subsection{}
%problem5.12
We want to know at one $n$ does this integral
\begin{eqnarray}
	\int_{.3}^{.5} \frac{1}{\sqrt{2 \pi .24/n}} e^{\frac{-(x-.4)^2}{2 \cdot .24/n}} \approx 0.9
\end{eqnarray}
A first pass going at intervals of 10 identifies the number being between 60 and 70 and a further pass going by one reveals that going from 64 to 65 breaks the .89 barrier.  Therefore $n=65$.

Note that in order to get that $n$ in the denominator, we are taking the relative mean and the relative standard deviation: since both quantities get divided by $n$, the square of the standard deviation ends up with an $n$ in the denominator.
\textbf{Answer verified}

\subsection{}
%problem5.13

\begin{eqnarray}
	\textbf{E}x = 60 \cdot .6 = 36 \\
	\textbf{D}x = 36 \cdot .4 = 14.4 \\
	p = \int_{30}^{\infty} \frac{1}{\sqrt{2 \pi 14.5}} e^{\frac{-(x-36)^2}{2 \cdot 14.4}} \approx 0.943077
\end{eqnarray}

\textbf{Answer verified}

\subsection{}
%problem5.14
Perform the integrals as suggested.  Why can you?  Well, it feels right, doesn't it?  I found it impossible to come up with other than an intuitive reason for why this is true: sorry dear reader.

\textbf{Answer verified-ish}

\subsection{}
%problem5.15

\textbf{SKIPPED}


\subsection{}
%problem5.16
Convolute the two distributions and the intended expression will come out: BOM!

\textbf{Answer verified-ish}

%%answer template
%\subsection{}
%%problem n.n
%
%
%\begin{equation}
%	
%\label{answern.n}
%\end{equation}
%\textbf{Answer [not] verified}





